% Options for packages loaded elsewhere
\PassOptionsToPackage{unicode}{hyperref}
\PassOptionsToPackage{hyphens}{url}
%
\documentclass[
]{article}
\usepackage{lmodern}
\usepackage{amssymb,amsmath}
\usepackage{ifxetex,ifluatex}
\ifnum 0\ifxetex 1\fi\ifluatex 1\fi=0 % if pdftex
  \usepackage[T1]{fontenc}
  \usepackage[utf8]{inputenc}
  \usepackage{textcomp} % provide euro and other symbols
\else % if luatex or xetex
  \usepackage{unicode-math}
  \defaultfontfeatures{Scale=MatchLowercase}
  \defaultfontfeatures[\rmfamily]{Ligatures=TeX,Scale=1}
\fi
% Use upquote if available, for straight quotes in verbatim environments
\IfFileExists{upquote.sty}{\usepackage{upquote}}{}
\IfFileExists{microtype.sty}{% use microtype if available
  \usepackage[]{microtype}
  \UseMicrotypeSet[protrusion]{basicmath} % disable protrusion for tt fonts
}{}
\makeatletter
\@ifundefined{KOMAClassName}{% if non-KOMA class
  \IfFileExists{parskip.sty}{%
    \usepackage{parskip}
  }{% else
    \setlength{\parindent}{0pt}
    \setlength{\parskip}{6pt plus 2pt minus 1pt}}
}{% if KOMA class
  \KOMAoptions{parskip=half}}
\makeatother
\usepackage{xcolor}
\IfFileExists{xurl.sty}{\usepackage{xurl}}{} % add URL line breaks if available
\IfFileExists{bookmark.sty}{\usepackage{bookmark}}{\usepackage{hyperref}}
\hypersetup{
  pdftitle={Prediction Assignment Writeup},
  pdfauthor={Kaaltho},
  hidelinks,
  pdfcreator={LaTeX via pandoc}}
\urlstyle{same} % disable monospaced font for URLs
\usepackage[margin=1in]{geometry}
\usepackage{color}
\usepackage{fancyvrb}
\newcommand{\VerbBar}{|}
\newcommand{\VERB}{\Verb[commandchars=\\\{\}]}
\DefineVerbatimEnvironment{Highlighting}{Verbatim}{commandchars=\\\{\}}
% Add ',fontsize=\small' for more characters per line
\usepackage{framed}
\definecolor{shadecolor}{RGB}{248,248,248}
\newenvironment{Shaded}{\begin{snugshade}}{\end{snugshade}}
\newcommand{\AlertTok}[1]{\textcolor[rgb]{0.94,0.16,0.16}{#1}}
\newcommand{\AnnotationTok}[1]{\textcolor[rgb]{0.56,0.35,0.01}{\textbf{\textit{#1}}}}
\newcommand{\AttributeTok}[1]{\textcolor[rgb]{0.77,0.63,0.00}{#1}}
\newcommand{\BaseNTok}[1]{\textcolor[rgb]{0.00,0.00,0.81}{#1}}
\newcommand{\BuiltInTok}[1]{#1}
\newcommand{\CharTok}[1]{\textcolor[rgb]{0.31,0.60,0.02}{#1}}
\newcommand{\CommentTok}[1]{\textcolor[rgb]{0.56,0.35,0.01}{\textit{#1}}}
\newcommand{\CommentVarTok}[1]{\textcolor[rgb]{0.56,0.35,0.01}{\textbf{\textit{#1}}}}
\newcommand{\ConstantTok}[1]{\textcolor[rgb]{0.00,0.00,0.00}{#1}}
\newcommand{\ControlFlowTok}[1]{\textcolor[rgb]{0.13,0.29,0.53}{\textbf{#1}}}
\newcommand{\DataTypeTok}[1]{\textcolor[rgb]{0.13,0.29,0.53}{#1}}
\newcommand{\DecValTok}[1]{\textcolor[rgb]{0.00,0.00,0.81}{#1}}
\newcommand{\DocumentationTok}[1]{\textcolor[rgb]{0.56,0.35,0.01}{\textbf{\textit{#1}}}}
\newcommand{\ErrorTok}[1]{\textcolor[rgb]{0.64,0.00,0.00}{\textbf{#1}}}
\newcommand{\ExtensionTok}[1]{#1}
\newcommand{\FloatTok}[1]{\textcolor[rgb]{0.00,0.00,0.81}{#1}}
\newcommand{\FunctionTok}[1]{\textcolor[rgb]{0.00,0.00,0.00}{#1}}
\newcommand{\ImportTok}[1]{#1}
\newcommand{\InformationTok}[1]{\textcolor[rgb]{0.56,0.35,0.01}{\textbf{\textit{#1}}}}
\newcommand{\KeywordTok}[1]{\textcolor[rgb]{0.13,0.29,0.53}{\textbf{#1}}}
\newcommand{\NormalTok}[1]{#1}
\newcommand{\OperatorTok}[1]{\textcolor[rgb]{0.81,0.36,0.00}{\textbf{#1}}}
\newcommand{\OtherTok}[1]{\textcolor[rgb]{0.56,0.35,0.01}{#1}}
\newcommand{\PreprocessorTok}[1]{\textcolor[rgb]{0.56,0.35,0.01}{\textit{#1}}}
\newcommand{\RegionMarkerTok}[1]{#1}
\newcommand{\SpecialCharTok}[1]{\textcolor[rgb]{0.00,0.00,0.00}{#1}}
\newcommand{\SpecialStringTok}[1]{\textcolor[rgb]{0.31,0.60,0.02}{#1}}
\newcommand{\StringTok}[1]{\textcolor[rgb]{0.31,0.60,0.02}{#1}}
\newcommand{\VariableTok}[1]{\textcolor[rgb]{0.00,0.00,0.00}{#1}}
\newcommand{\VerbatimStringTok}[1]{\textcolor[rgb]{0.31,0.60,0.02}{#1}}
\newcommand{\WarningTok}[1]{\textcolor[rgb]{0.56,0.35,0.01}{\textbf{\textit{#1}}}}
\usepackage{graphicx,grffile}
\makeatletter
\def\maxwidth{\ifdim\Gin@nat@width>\linewidth\linewidth\else\Gin@nat@width\fi}
\def\maxheight{\ifdim\Gin@nat@height>\textheight\textheight\else\Gin@nat@height\fi}
\makeatother
% Scale images if necessary, so that they will not overflow the page
% margins by default, and it is still possible to overwrite the defaults
% using explicit options in \includegraphics[width, height, ...]{}
\setkeys{Gin}{width=\maxwidth,height=\maxheight,keepaspectratio}
% Set default figure placement to htbp
\makeatletter
\def\fps@figure{htbp}
\makeatother
\setlength{\emergencystretch}{3em} % prevent overfull lines
\providecommand{\tightlist}{%
  \setlength{\itemsep}{0pt}\setlength{\parskip}{0pt}}
\setcounter{secnumdepth}{-\maxdimen} % remove section numbering

\title{Prediction Assignment Writeup}
\author{Kaaltho}
\date{26/12/2020}

\begin{document}
\maketitle

\hypertarget{introduction}{%
\subsection{Introduction}\label{introduction}}

The goal of your project is to predict the manner in which they did the
exercise. This is the ``classe'' variable in the training set. You may
use any of the other variables to predict with. You should create a
report describing how you built your model, how you used cross
validation, what you think the expected out of sample error is, and why
you made the choices you did. You will also use your prediction model to
predict 20 different test cases.

\textbf{\emph{Background}}

Using devices such as Jawbone Up, Nike FuelBand, and Fitbit it is now
possible to collect a large amount of data about personal activity
relatively inexpensively. These type of devices are part of the
quantified self movement -- a group of enthusiasts who take measurements
about themselves regularly to improve their health, to find patterns in
their behavior, or because they are tech geeks. One thing that people
regularly do is quantify how much of a particular activity they do, but
they rarely quantify how well they do it. In this project, your goal
will be to use data from accelerometers on the belt, forearm, arm, and
dumbell of 6 participants. They were asked to perform barbell lifts
correctly and incorrectly in 5 different ways. More information is
available from the website here:
\url{http://web.archive.org/web/20161224072740/http:/groupware.les.inf.puc-rio.br/har}
(see the section on the Weight Lifting Exercise Dataset).

\begin{Shaded}
\begin{Highlighting}[]
\CommentTok{#Loading libraries}
\KeywordTok{library}\NormalTok{(caret)}
\end{Highlighting}
\end{Shaded}

\begin{verbatim}
## Loading required package: lattice
\end{verbatim}

\begin{verbatim}
## Loading required package: ggplot2
\end{verbatim}

\begin{Shaded}
\begin{Highlighting}[]
\KeywordTok{library}\NormalTok{(ggplot2)}
\KeywordTok{library}\NormalTok{(RColorBrewer)}
\KeywordTok{library}\NormalTok{(lattice)}
\KeywordTok{library}\NormalTok{(randomForest)}
\end{Highlighting}
\end{Shaded}

\begin{verbatim}
## randomForest 4.6-14
\end{verbatim}

\begin{verbatim}
## Type rfNews() to see new features/changes/bug fixes.
\end{verbatim}

\begin{verbatim}
## 
## Attaching package: 'randomForest'
\end{verbatim}

\begin{verbatim}
## The following object is masked from 'package:ggplot2':
## 
##     margin
\end{verbatim}

\begin{Shaded}
\begin{Highlighting}[]
\KeywordTok{library}\NormalTok{(rpart)}
\KeywordTok{library}\NormalTok{(rpart.plot)}
\KeywordTok{library}\NormalTok{(corrplot)}
\end{Highlighting}
\end{Shaded}

\begin{verbatim}
## corrplot 0.84 loaded
\end{verbatim}

\begin{Shaded}
\begin{Highlighting}[]
\KeywordTok{library}\NormalTok{(rattle)}
\end{Highlighting}
\end{Shaded}

\begin{verbatim}
## Loading required package: tibble
\end{verbatim}

\begin{verbatim}
## Loading required package: bitops
\end{verbatim}

\begin{verbatim}
## Rattle: A free graphical interface for data science with R.
## Versión 5.4.0 Copyright (c) 2006-2020 Togaware Pty Ltd.
## Escriba 'rattle()' para agitar, sacudir y  rotar sus datos.
\end{verbatim}

\begin{verbatim}
## 
## Attaching package: 'rattle'
\end{verbatim}

\begin{verbatim}
## The following object is masked from 'package:randomForest':
## 
##     importance
\end{verbatim}

\begin{Shaded}
\begin{Highlighting}[]
\KeywordTok{set.seed}\NormalTok{(}\DecValTok{1813}\NormalTok{)}

\CommentTok{#Loading data}
\CommentTok{#Set the working directory to your own and download the files to it}
\KeywordTok{setwd}\NormalTok{(}\StringTok{"F:/R codes/Projects/datasciencecoursera/Practical Machine Learning"}\NormalTok{)}

\CommentTok{# File download}
\NormalTok{ulrtraining <-}\StringTok{ "http://d396qusza40orc.cloudfront.net/predmachlearn/pml-training.csv"}
\NormalTok{ulrtest <-}\StringTok{ "http://d396qusza40orc.cloudfront.net/predmachlearn/pml-testing.csv"}
\NormalTok{dirfile1 <-}\StringTok{ "F:/R codes/Projects/datasciencecoursera/Practical Machine Learning/pml-training.csv"}
\NormalTok{dirfile2 <-}\StringTok{ "F:/R codes/Projects/datasciencecoursera/Practical Machine Learning/pml-testing.csv"}

\KeywordTok{download.file}\NormalTok{(ulrtraining, }\DataTypeTok{destfile =}\NormalTok{ dirfile1)}
\KeywordTok{download.file}\NormalTok{(ulrtest, }\DataTypeTok{destfile =}\NormalTok{dirfile2)}

\CommentTok{#Assigning the dataset to a varible}
\NormalTok{training <-}\StringTok{ }\KeywordTok{read.csv}\NormalTok{(dirfile1, }\DataTypeTok{strip.white =} \OtherTok{TRUE}\NormalTok{, }\DataTypeTok{na.strings =} \KeywordTok{c}\NormalTok{(}\StringTok{"NA"}\NormalTok{,}\StringTok{""}\NormalTok{))}
\NormalTok{test<-}\StringTok{ }\KeywordTok{read.csv}\NormalTok{(dirfile2,  }\DataTypeTok{strip.white =} \OtherTok{TRUE}\NormalTok{, }\DataTypeTok{na.strings =} \KeywordTok{c}\NormalTok{(}\StringTok{"NA"}\NormalTok{,}\StringTok{""}\NormalTok{))}

\CommentTok{#To have an idea of the size of the data set}
\NormalTok{dim1 <-}\StringTok{ }\KeywordTok{dim}\NormalTok{(training)}
\NormalTok{dim2 <-}\StringTok{ }\KeywordTok{dim}\NormalTok{(test)}
\end{Highlighting}
\end{Shaded}

This is an R Markdown document. Markdown is a simple formatting syntax
for authoring HTML, PDF, and MS Word documents. For more details on
using R Markdown see \url{http://rmarkdown.rstudio.com}.

When you click the \textbf{Knit} button a document will be generated
that includes both content as well as the output of any embedded R code
chunks within the document. You can embed an R code chunk like this:

\begin{Shaded}
\begin{Highlighting}[]
\KeywordTok{summary}\NormalTok{(cars)}
\end{Highlighting}
\end{Shaded}

\begin{verbatim}
##      speed           dist       
##  Min.   : 4.0   Min.   :  2.00  
##  1st Qu.:12.0   1st Qu.: 26.00  
##  Median :15.0   Median : 36.00  
##  Mean   :15.4   Mean   : 42.98  
##  3rd Qu.:19.0   3rd Qu.: 56.00  
##  Max.   :25.0   Max.   :120.00
\end{verbatim}

\hypertarget{including-plots}{%
\subsection{Including Plots}\label{including-plots}}

You can also embed plots, for example:

\includegraphics{Prediction-Assignment-Writeup_files/figure-latex/pressure-1.pdf}

Note that the \texttt{echo\ =\ FALSE} parameter was added to the code
chunk to prevent printing of the R code that generated the plot.

\end{document}
