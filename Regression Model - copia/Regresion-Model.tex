% Options for packages loaded elsewhere
\PassOptionsToPackage{unicode}{hyperref}
\PassOptionsToPackage{hyphens}{url}
%
\documentclass[
]{article}
\usepackage{lmodern}
\usepackage{amssymb,amsmath}
\usepackage{ifxetex,ifluatex}
\ifnum 0\ifxetex 1\fi\ifluatex 1\fi=0 % if pdftex
  \usepackage[T1]{fontenc}
  \usepackage[utf8]{inputenc}
  \usepackage{textcomp} % provide euro and other symbols
\else % if luatex or xetex
  \usepackage{unicode-math}
  \defaultfontfeatures{Scale=MatchLowercase}
  \defaultfontfeatures[\rmfamily]{Ligatures=TeX,Scale=1}
\fi
% Use upquote if available, for straight quotes in verbatim environments
\IfFileExists{upquote.sty}{\usepackage{upquote}}{}
\IfFileExists{microtype.sty}{% use microtype if available
  \usepackage[]{microtype}
  \UseMicrotypeSet[protrusion]{basicmath} % disable protrusion for tt fonts
}{}
\makeatletter
\@ifundefined{KOMAClassName}{% if non-KOMA class
  \IfFileExists{parskip.sty}{%
    \usepackage{parskip}
  }{% else
    \setlength{\parindent}{0pt}
    \setlength{\parskip}{6pt plus 2pt minus 1pt}}
}{% if KOMA class
  \KOMAoptions{parskip=half}}
\makeatother
\usepackage{xcolor}
\IfFileExists{xurl.sty}{\usepackage{xurl}}{} % add URL line breaks if available
\IfFileExists{bookmark.sty}{\usepackage{bookmark}}{\usepackage{hyperref}}
\hypersetup{
  pdftitle={Regression},
  pdfauthor={Kaaltho},
  hidelinks,
  pdfcreator={LaTeX via pandoc}}
\urlstyle{same} % disable monospaced font for URLs
\usepackage[margin=1in]{geometry}
\usepackage{color}
\usepackage{fancyvrb}
\newcommand{\VerbBar}{|}
\newcommand{\VERB}{\Verb[commandchars=\\\{\}]}
\DefineVerbatimEnvironment{Highlighting}{Verbatim}{commandchars=\\\{\}}
% Add ',fontsize=\small' for more characters per line
\usepackage{framed}
\definecolor{shadecolor}{RGB}{248,248,248}
\newenvironment{Shaded}{\begin{snugshade}}{\end{snugshade}}
\newcommand{\AlertTok}[1]{\textcolor[rgb]{0.94,0.16,0.16}{#1}}
\newcommand{\AnnotationTok}[1]{\textcolor[rgb]{0.56,0.35,0.01}{\textbf{\textit{#1}}}}
\newcommand{\AttributeTok}[1]{\textcolor[rgb]{0.77,0.63,0.00}{#1}}
\newcommand{\BaseNTok}[1]{\textcolor[rgb]{0.00,0.00,0.81}{#1}}
\newcommand{\BuiltInTok}[1]{#1}
\newcommand{\CharTok}[1]{\textcolor[rgb]{0.31,0.60,0.02}{#1}}
\newcommand{\CommentTok}[1]{\textcolor[rgb]{0.56,0.35,0.01}{\textit{#1}}}
\newcommand{\CommentVarTok}[1]{\textcolor[rgb]{0.56,0.35,0.01}{\textbf{\textit{#1}}}}
\newcommand{\ConstantTok}[1]{\textcolor[rgb]{0.00,0.00,0.00}{#1}}
\newcommand{\ControlFlowTok}[1]{\textcolor[rgb]{0.13,0.29,0.53}{\textbf{#1}}}
\newcommand{\DataTypeTok}[1]{\textcolor[rgb]{0.13,0.29,0.53}{#1}}
\newcommand{\DecValTok}[1]{\textcolor[rgb]{0.00,0.00,0.81}{#1}}
\newcommand{\DocumentationTok}[1]{\textcolor[rgb]{0.56,0.35,0.01}{\textbf{\textit{#1}}}}
\newcommand{\ErrorTok}[1]{\textcolor[rgb]{0.64,0.00,0.00}{\textbf{#1}}}
\newcommand{\ExtensionTok}[1]{#1}
\newcommand{\FloatTok}[1]{\textcolor[rgb]{0.00,0.00,0.81}{#1}}
\newcommand{\FunctionTok}[1]{\textcolor[rgb]{0.00,0.00,0.00}{#1}}
\newcommand{\ImportTok}[1]{#1}
\newcommand{\InformationTok}[1]{\textcolor[rgb]{0.56,0.35,0.01}{\textbf{\textit{#1}}}}
\newcommand{\KeywordTok}[1]{\textcolor[rgb]{0.13,0.29,0.53}{\textbf{#1}}}
\newcommand{\NormalTok}[1]{#1}
\newcommand{\OperatorTok}[1]{\textcolor[rgb]{0.81,0.36,0.00}{\textbf{#1}}}
\newcommand{\OtherTok}[1]{\textcolor[rgb]{0.56,0.35,0.01}{#1}}
\newcommand{\PreprocessorTok}[1]{\textcolor[rgb]{0.56,0.35,0.01}{\textit{#1}}}
\newcommand{\RegionMarkerTok}[1]{#1}
\newcommand{\SpecialCharTok}[1]{\textcolor[rgb]{0.00,0.00,0.00}{#1}}
\newcommand{\SpecialStringTok}[1]{\textcolor[rgb]{0.31,0.60,0.02}{#1}}
\newcommand{\StringTok}[1]{\textcolor[rgb]{0.31,0.60,0.02}{#1}}
\newcommand{\VariableTok}[1]{\textcolor[rgb]{0.00,0.00,0.00}{#1}}
\newcommand{\VerbatimStringTok}[1]{\textcolor[rgb]{0.31,0.60,0.02}{#1}}
\newcommand{\WarningTok}[1]{\textcolor[rgb]{0.56,0.35,0.01}{\textbf{\textit{#1}}}}
\usepackage{graphicx,grffile}
\makeatletter
\def\maxwidth{\ifdim\Gin@nat@width>\linewidth\linewidth\else\Gin@nat@width\fi}
\def\maxheight{\ifdim\Gin@nat@height>\textheight\textheight\else\Gin@nat@height\fi}
\makeatother
% Scale images if necessary, so that they will not overflow the page
% margins by default, and it is still possible to overwrite the defaults
% using explicit options in \includegraphics[width, height, ...]{}
\setkeys{Gin}{width=\maxwidth,height=\maxheight,keepaspectratio}
% Set default figure placement to htbp
\makeatletter
\def\fps@figure{htbp}
\makeatother
\setlength{\emergencystretch}{3em} % prevent overfull lines
\providecommand{\tightlist}{%
  \setlength{\itemsep}{0pt}\setlength{\parskip}{0pt}}
\setcounter{secnumdepth}{-\maxdimen} % remove section numbering

\title{Regression}
\author{Kaaltho}
\date{26/12/2020}

\begin{document}
\maketitle

\hypertarget{executive-summary}{%
\subsection{Executive Summary}\label{executive-summary}}

One of the features that are important in cars is miles per gallon (MPG)
consume, every vehicle has his own specifications, one of the most
frequent question ask by buyers about MPG consume is: What is better a
manual transmission or automatic transmission. This analysis provide an
aswer to the following questions:

\begin{enumerate}
\def\labelenumi{\arabic{enumi}.}
\tightlist
\item
  ``Is an automatic or manual transmission better for MPG''
\item
  ``Quantify the MPG difference between automatic and manual
  transmissions''
\end{enumerate}

\hypertarget{exploratory-data-analysis}{%
\subsubsection{Exploratory data
analysis}\label{exploratory-data-analysis}}

The data used to this analysis is from Motor Trend US magazine from
1974, which give some metric for the fuel consumption and performance of
32 different automobiles

\begin{Shaded}
\begin{Highlighting}[]
\CommentTok{#Oppening libraries}
\KeywordTok{library}\NormalTok{(ggplot2)}
\KeywordTok{library}\NormalTok{(cowplot)}

\CommentTok{#Loading the data:}
\KeywordTok{data}\NormalTok{(mtcars)}

\CommentTok{#changing the type of variables for convenience in order to do calculations.}
\NormalTok{mtcars2 <-}\StringTok{ }\NormalTok{mtcars}

\NormalTok{cols <-}\StringTok{ }\KeywordTok{c}\NormalTok{(}\StringTok{"cyl"}\NormalTok{, }\StringTok{"vs"}\NormalTok{, }\StringTok{"gear"}\NormalTok{, }\StringTok{"carb"}\NormalTok{)}
\NormalTok{mtcars2[cols] <-}\StringTok{ }\KeywordTok{lapply}\NormalTok{(mtcars2[cols], factor)}

\CommentTok{#new column name transmissions for better plotting understanding}
\NormalTok{mtcars2}\OperatorTok{$}\NormalTok{transmission <-}\StringTok{ }\KeywordTok{factor}\NormalTok{(mtcars}\OperatorTok{$}\NormalTok{am, }\DataTypeTok{labels=}\KeywordTok{c}\NormalTok{(}\StringTok{"Automatic"}\NormalTok{,}\StringTok{"Manual"}\NormalTok{))}

\CommentTok{#mean fo automatic and manual transmission}
\NormalTok{mean.trans <-}\StringTok{ }\KeywordTok{aggregate}\NormalTok{(}\DataTypeTok{data=}\NormalTok{mtcars2, mpg}\OperatorTok{~}\NormalTok{transmission, mean)}


\CommentTok{#Visualization of the data}
\NormalTok{plot1 <-}\StringTok{ }\KeywordTok{ggplot}\NormalTok{(mtcars2, }\KeywordTok{aes}\NormalTok{(}\DataTypeTok{x=}\NormalTok{transmission, }\DataTypeTok{y=}\NormalTok{mpg)) }\OperatorTok{+}
\StringTok{  }\KeywordTok{geom_boxplot}\NormalTok{(}\DataTypeTok{colour=}\StringTok{"darkviolet"}\NormalTok{, }\DataTypeTok{fill=}\StringTok{"skyblue"}\NormalTok{)}\OperatorTok{+}\StringTok{ }
\StringTok{  }\KeywordTok{guides}\NormalTok{(}\DataTypeTok{fill=}\OtherTok{FALSE}\NormalTok{) }\OperatorTok{+}
\StringTok{  }\KeywordTok{labs}\NormalTok{(}\DataTypeTok{x=}\StringTok{"Transmission Type"}\NormalTok{, }\DataTypeTok{y=}\StringTok{"Miles per Gallon"}\NormalTok{, }\DataTypeTok{title=}\NormalTok{(}\StringTok{"Comparisson bettewn Manual and Automatic Transmission"}\NormalTok{))}\OperatorTok{+}
\StringTok{  }\KeywordTok{theme}\NormalTok{(}\DataTypeTok{plot.title =} \KeywordTok{element_text}\NormalTok{(}\DataTypeTok{hjust =} \FloatTok{0.5}\NormalTok{))}
\NormalTok{plot1}
\end{Highlighting}
\end{Shaded}

\includegraphics{Regresion-Model_files/figure-latex/loading-1.pdf}

It is possible to make initial assumptions, for instance, in the figure
there is a clear difference between Automatic transmission and Manual
transmission it seem that Automatic transmission is more efficient when
it comes to fuel consumption with a mean of 17.15 Miles per galon ,
while Manual transmission show a consumption of 24.39 Miles per galon.

\hypertarget{applying-regresion-models}{%
\subsubsection{Applying Regresion
models}\label{applying-regresion-models}}

Considering the previous results it is possible to say that there is a
difference of 7.25 MPG in relation to the type of transmission, being
Automatic transmission the one that consumes less fuel. This assumption
could be analyzed through a simple linear regression.

\begin{Shaded}
\begin{Highlighting}[]
\CommentTok{#linear regression}
\NormalTok{linear.regression <-}\StringTok{ }\KeywordTok{lm}\NormalTok{(mpg }\OperatorTok{~}\StringTok{ }\KeywordTok{factor}\NormalTok{(am), }\DataTypeTok{data=}\NormalTok{mtcars2)}
\KeywordTok{summary}\NormalTok{(linear.regression)}
\end{Highlighting}
\end{Shaded}

\begin{verbatim}
## 
## Call:
## lm(formula = mpg ~ factor(am), data = mtcars2)
## 
## Residuals:
##     Min      1Q  Median      3Q     Max 
## -9.3923 -3.0923 -0.2974  3.2439  9.5077 
## 
## Coefficients:
##             Estimate Std. Error t value Pr(>|t|)    
## (Intercept)   17.147      1.125  15.247 1.13e-15 ***
## factor(am)1    7.245      1.764   4.106 0.000285 ***
## ---
## Signif. codes:  0 '***' 0.001 '**' 0.01 '*' 0.05 '.' 0.1 ' ' 1
## 
## Residual standard error: 4.902 on 30 degrees of freedom
## Multiple R-squared:  0.3598, Adjusted R-squared:  0.3385 
## F-statistic: 16.86 on 1 and 30 DF,  p-value: 0.000285
\end{verbatim}

From the calculation p-value is 0.000285, the hypothesis is not
rejected, seeing the R-squared value can be said that a third of the
variance (36\%) could be attributed to just transmission type. To
further analysis the next step is to do an Analysis of Variance.

\begin{Shaded}
\begin{Highlighting}[]
\CommentTok{#variance analysis}
\NormalTok{mt.variance <-}\StringTok{ }\KeywordTok{aov}\NormalTok{(mpg }\OperatorTok{~}\StringTok{ }\NormalTok{., }\DataTypeTok{data =}\NormalTok{ mtcars2)}
\KeywordTok{summary}\NormalTok{(mt.variance)}
\end{Highlighting}
\end{Shaded}

\begin{verbatim}
##             Df Sum Sq Mean Sq F value   Pr(>F)    
## cyl          2  824.8   412.4  51.377 1.94e-07 ***
## disp         1   57.6    57.6   7.181   0.0171 *  
## hp           1   18.5    18.5   2.305   0.1497    
## drat         1   11.9    11.9   1.484   0.2419    
## wt           1   55.8    55.8   6.950   0.0187 *  
## qsec         1    1.5     1.5   0.190   0.6692    
## vs           1    0.3     0.3   0.038   0.8488    
## am           1   16.6    16.6   2.064   0.1714    
## gear         2    5.0     2.5   0.313   0.7361    
## carb         5   13.6     2.7   0.339   0.8814    
## Residuals   15  120.4     8.0                     
## ---
## Signif. codes:  0 '***' 0.001 '**' 0.01 '*' 0.05 '.' 0.1 ' ' 1
\end{verbatim}

Some of the variables that could affect the variance in greater way are
cyl, disp and wt. This variables are going to be used in the next
analysis to see how those affect MPG.

\begin{Shaded}
\begin{Highlighting}[]
\CommentTok{#Multivariable analysis}
\NormalTok{mt.multi <-}\StringTok{ }\KeywordTok{lm}\NormalTok{(mpg }\OperatorTok{~}\StringTok{ }\NormalTok{am }\OperatorTok{+}\StringTok{ }\NormalTok{cyl }\OperatorTok{+}\StringTok{ }\NormalTok{disp }\OperatorTok{+}\StringTok{ }\NormalTok{wt, }\DataTypeTok{data =}\NormalTok{ mtcars2)}
\KeywordTok{summary}\NormalTok{(mt.multi)}
\end{Highlighting}
\end{Shaded}

\begin{verbatim}
## 
## Call:
## lm(formula = mpg ~ am + cyl + disp + wt, data = mtcars2)
## 
## Residuals:
##     Min      1Q  Median      3Q     Max 
## -4.5029 -1.2829 -0.4825  1.4954  5.7889 
## 
## Coefficients:
##              Estimate Std. Error t value Pr(>|t|)    
## (Intercept) 33.816067   2.914272  11.604 8.79e-12 ***
## am           0.141212   1.326751   0.106  0.91605    
## cyl6        -4.304782   1.492355  -2.885  0.00777 ** 
## cyl8        -6.318406   2.647658  -2.386  0.02458 *  
## disp         0.001632   0.013757   0.119  0.90647    
## wt          -3.249176   1.249098  -2.601  0.01513 *  
## ---
## Signif. codes:  0 '***' 0.001 '**' 0.01 '*' 0.05 '.' 0.1 ' ' 1
## 
## Residual standard error: 2.652 on 26 degrees of freedom
## Multiple R-squared:  0.8376, Adjusted R-squared:  0.8064 
## F-statistic: 26.82 on 5 and 26 DF,  p-value: 1.73e-09
\end{verbatim}

This model show that 83\% of the variability can be attribute to the
cyl, disp, hp and wt variables, furthermore p-value for cy16 and cy18
are below 0.5, denoting that those variables could have a fair share of
influence over the variability of Miles per Gallon so we can say that
there is a correlation between those variables and transmission type and
Miles per Gallon.

\textbf{\emph{MPG difference between automatic and manual transmission}}

\begin{Shaded}
\begin{Highlighting}[]
\NormalTok{modelfit <-}\StringTok{ }\KeywordTok{lm}\NormalTok{(mpg}\OperatorTok{~}\NormalTok{., }\DataTypeTok{data=}\NormalTok{mtcars2)}
\NormalTok{stepmodel <-}\StringTok{ }\KeywordTok{step}\NormalTok{(modelfit)}
\KeywordTok{summary}\NormalTok{(stepmodel)}\OperatorTok{$}\NormalTok{coeff}
\end{Highlighting}
\end{Shaded}

Finally the diferrence between automatic and manual transmissions is
1.81 MPG.

\hypertarget{appendix}{%
\subsubsection{Appendix}\label{appendix}}

Residual Analysis

The ``Residuals vs Fitted'' plot shows that the residuals are
homoscedastic. Also can be seen that they are normally distributed.

\begin{Shaded}
\begin{Highlighting}[]
\KeywordTok{par}\NormalTok{(}\DataTypeTok{mfrow =} \KeywordTok{c}\NormalTok{(}\DecValTok{2}\NormalTok{,}\DecValTok{2}\NormalTok{))}
\KeywordTok{plot}\NormalTok{(mt.multi)}
\end{Highlighting}
\end{Shaded}

\includegraphics{Regresion-Model_files/figure-latex/residual-1.pdf}

\end{document}
